\begin{abstract}
\noindent
Generative Adversarial Networks (GANs) have gained much research attention ever since their initial invention. They have been shown to produce impressive results for image generation with great success. Yet, there has been no work or only a little done in using images generated by GAN for enhancing classification tasks especially in areas of character recognition. This thesis attempt to explore, in the context of the images of nameplates/steel type plates attached to power supply equipment, whether it is possible to improve their character recognition accuracy using GAN generated image. Nowadays Optical Character Recognition (OCR) algorithms are powerful enough to read any type of text such as printed, written or even embossed information present in the images. But still, the greatest difficulty in adopting OCR is that every use case is unique and requires appropriate preprocessing methods. In this study, we have approached the problem as a translation of noisy images into clear images for improving character recognition accuracy. A carefully designed pipeline is proposed that includes an image-to-image translation GAN model to optimize the noisy images. Then these optimized images are fed into the OCR engine for performing accurate character recognition. We investigated three kinds of GAN, one that trains in a supervised manner (pix2pix), one in an unsupervised manner (CycleGAN), and lastly in a semi-supervised manner (FactorGAN). The experimental results show that pix2pix GAN performs well in eliminating the noisy factors present in the image such as rust, corrosion, difficult lighting conditions, reflections or shadows and enhance the contrast between the text and background. Most significantly, it can be seen that the character recognition of steel type plates has been increased by 37 \% for the proposed OCR using GAN generated images.
\newline
\newline
\textbf{Keywords:} nameplates/steel type plates, GAN, image-to-image translation, pix2pix, CycleGAN, FactorGAN, OCR.

\end{abstract}

\renewcommand{\abstractname}{Acknowledgements}
\begin{abstract}
I would like to thank and express my gratitude to Prof. Dr. Dr. Lars Schmidt-Thieme, Lukas Brinkmeyer, Regina Kessler and Thomas Kessler for helping me to bring this work to fruition. 
\newline

	I am grateful to Regina Kessler and Genie Enterprise Inc for the warm welcome and financial support all through the development of this thesis. 
\newline
 
	To Thomas Kessler, I wish to express my gratitude for the many stimulating conversations which led to the inception of the core idea behind the thesis. He was also instrumental in giving constructive criticism and feedback, as well as technical support during this period.
\newline

	I also wish to particularly thank Lukas Brinkmeyer for his constant supportive comments, valuable insights and knowledge which helped in shaping the thesis.
\newline

	Lastly, I am thankful to my friends and family, especially my parents, for their perennial support and encouragement.
\newline

	A special thanks to Adaobi Onyeakagbu for providing her feedback on the report.
\newline
 \end{abstract}